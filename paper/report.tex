\documentclass[12pt]{article}
\usepackage{amsmath,amssymb,amsthm}
\usepackage{graphicx}
\usepackage[ruled,vlined,linesnumbered]{algorithm2e}

\title{Optimizing Volunteer Shifts for Dance Weekends}
\author{John Jeng, Samvel Stepanyan}
\date{January 22nd, 2015}
\pagestyle{myheadings}
\markright{}

% Custom commands
\newcommand{\sumleft}{\sum\limits_{i=1}^n i^3}
\newcommand{\sumright}{\left(\sum\limits_{i=1}^n i \right)^2}
\newcommand{\Z}{\mathbb{Z}}
\newcommand{\C}{\mathbb{C}}
\newcommand{\pem}{$ (\pi, E) $~}

\newtheorem{PF}{Proof}
\theoremstyle{definition}
\newtheorem{Def}{Definition}
\newtheorem{Thm}{Theorem}
\newtheorem{lemma}{Lemma}

\begin{document}
%Begin Formalities========
%=========================
\maketitle
\begin{abstract}
	The Savoy Swing Club (SSC) in Seattle, WA is a non-profit that organizes dance weekends for the general public and has outreach programs to middle schools and high schools in the greater Seattle area. All events require volunteers to run the event. Volunteer scheduling is usually done by hand, but becomes harder with larger events and does not take into account many variables that could be considered such as volunteer time preferences and minimize the number of volunteers used. Our objective was to provide a program which would take a pool of volunteers and set of shifts and jobs, and return a schedule of volunteers that optimized for work balance, number of volunteers, ``choas'', and shift preferences.
\end{abstract}
\newpage
\tableofcontents
\newpage
%End Formalities=========
%========================

\section{Introduction}

Savoy Swing Club (SSC) is local non-profit organization that currently seeks to teach swing dancing to middle and high school kids in the greater Seattle area.
They are also heavily involved with 2 “dance weekends”: Seattle Lindy Exchange (SLX) and Killerdiller Weekend (KDW).

The format of dance weekends includes live bands, classe, performances, and competitions. All of these activities requires volunteers to make the weekend go smoothly.
Each volunteer has a set of capabilities (sound, driving, door management, etc) and preferences for which hours (and possibly how many) they are volunteering.
Volunteers are compensated based on the number of hours they work and the type of work.
For example, 8 hours of volunteering might mean “full pass” (access to all classes and evening dances) while fewer hours of volunteering might just mean a “dance pass” (only access to evening dances) or a “partial full pass” (eg. classes and evening dances on Saturday only).
A technician however, might receive a full weekend pass and additional cash compensation for just 6 hours of work.

Given hourly requirements for various jobs and a set of volunteers, SSC would like to minimize the total compensation payout while adhering as much as possible to the volunteers’ preferences. As a side note, SSC would like to pay out in passes as much as possible over cash since the realized expense is much lower than raw cash expenses.

\section{Problem Description}
Our focus is on ``full-featured'' dance events where there are classes, evening dances, performances, and competitions

\section{Simplifying Assumptions}

\subsection{Notes from Data}

\section{Mathematical Formulation}

\section{Primary Algorithm}

\section{Results}

\begin{thebibliography}{9}
\bibitem{Fed87}
	Greg N Federickson.
	Fast algorithms for shortest paths in planar graphs, with applications.
	SIAM Journal on Computing, 16(6):1004-1022,
	1987

\bibitem{Klein13}
	Monika R. Henzinger, Philip Klein, Satish Rao, Sairam Subramanian.
	Faster Shortest-Path Algorithms for Planar Graphs
	Journal of Computer and System Sciences Article NO. SS971493
	1997

\end{thebibliography}

\end{document}
