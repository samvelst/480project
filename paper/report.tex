\documentclass[12pt]{article}
\usepackage{amsmath,amssymb,amsthm}
\usepackage{graphicx}
\usepackage[ruled,vlined,linesnumbered]{algorithm2e}

\title{Optimizing Volunteer Shifts for Dance Weekends}
\author{John Jeng, Samvel Stepanyan}
\date{January 22nd, 2015}
\pagestyle{myheadings}
\markright{}

% Custom commands
\newcommand{\sumleft}{\sum\limits_{i=1}^n i^3}
\newcommand{\sumright}{\left(\sum\limits_{i=1}^n i \right)^2}
\newcommand{\Z}{\mathbb{Z}}
\newcommand{\C}{\mathbb{C}}
\newcommand{\pem}{$ (\pi, E) $~}

\newtheorem{PF}{Proof}
\theoremstyle{definition}
\newtheorem{Def}{Definition}
\newtheorem{Thm}{Theorem}
\newtheorem{lemma}{Lemma}

\begin{document}
%Begin Formalities========
%=========================
\maketitle
\begin{abstract}
	The Savoy Swing Club (SSC) in Seattle, WA is a non-profit that organizes dance weekends for the general public and has outreach programs to middle schools and high schools in the greater Seattle area. All events require volunteers to run the event. Volunteer scheduling is usually done by hand, but becomes harder with larger events and does not take into account many variables that could be considered such as volunteer time preferences and minimize the number of volunteers used. Our objective was to provide a program which would take a pool of volunteers and set of shifts and jobs, and return a schedule of volunteers that optimized for work balance, number of volunteers, ``choas'', and shift preferences.
\end{abstract}
\newpage
\tableofcontents
\newpage
%End Formalities=========
%========================

\section{Problem Description}

Savoy Swing Club (SSC) is local non-profit organization that currently seeks to teach swing dancing to middle and high school kids in the greater Seattle area.
They are also heavily involved with 2 “dance weekends”: Seattle Lindy Exchange (SLX) and Killerdiller Weekend (KDW).

Our focus is on ``full-featured'' dance events where there are classes, evening dances, performances, and competitions.
At each of these events, volunteers are required to make the weekend go smoothly.

Each volunteer has different jobs they are able (or allowed) to perform and preferences for which hours and possibly how many they are volunteering.
Volunteers are compensated based on the number of hours they work and the type of work.
For example, 8 hours of volunteering might mean “full pass” (access to all classes and evening dances) while fewer hours of volunteering might just mean a “dance pass” (only access to evening dances) or a “partial full pass” (eg. classes and evening dances on Saturday only).
A technician however, might receive a full weekend pass and additional cash compensation for just 6 hours of work.

Given hourly requirements for various jobs and a set of volunteers, SSC would like to minimize the total compensation payout while adhering as much as possible to the volunteers’ preferences.

One of our primary objectives was to allow volunteers to input times that they would be unavailable to work.
Typically, volunteers do not know exactly what hours they will be working until at most a few days before the event.
This means that volunteers must commit their entire weekends even though they may only be working 6 to 8 hours over 2 days.



\section{Simplifying Assumptions}

\subsection{Notes from Data}

\section{Mathematical Formulation}

\section{Primary Algorithm}

\section{Results}

\begin{thebibliography}{9}
\bibitem{Fed87}
	Greg N Federickson.
	Fast algorithms for shortest paths in planar graphs, with applications.
	SIAM Journal on Computing, 16(6):1004-1022,
	1987

\bibitem{Klein13}
	Monika R. Henzinger, Philip Klein, Satish Rao, Sairam Subramanian.
	Faster Shortest-Path Algorithms for Planar Graphs
	Journal of Computer and System Sciences Article NO. SS971493
	1997

\end{thebibliography}

\end{document}
